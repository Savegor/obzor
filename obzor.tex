\documentclass[a4paper, 12pt]{report}
\usepackage[T2A]{fontenc}
\usepackage[english, russian]{babel}
\usepackage{graphicx}
\graphicspath{{./images/}}
\usepackage[utf8]{inputenc}
\usepackage[backend=biber,bibencoding=utf8,sorting=nty,maxcitenames=2,style=numeric-comp]{biblatex}
\addbibresource{bibliography.bib}

\usepackage{color} %% это для отображения цвета в коде
\usepackage{listings} %% собственно, это и есть пакет listings
\usepackage{caption}
\DeclareCaptionFont{white}{\color{white}} %% это сделает текст заголовка белым
%% код ниже нарисует серую рамочку вокруг заголовка кода.
\DeclareCaptionFormat{listing}{\colorbox{gray}{\parbox{\textwidth}{#1#2#3}}}
\captionsetup[lstlisting]{format=listing,labelfont=white,textfont=white}

\begin{document}


\lstset{ %
language=C,                 % выбор языка для подсветки (здесь это С)
basicstyle=\small\sffamily, % размер и начертание шрифта для подсветки кода
numbers=left,               % где поставить нумерацию строк (слева\справа)
numberstyle=\tiny,           % размер шрифта для номеров строк
stepnumber=1,                   % размер шага между двумя номерами строк
numbersep=5pt,                % как далеко отстоят номера строк от подсвечиваемого кода
backgroundcolor=\color{white}, % цвет фона подсветки - используем \usepackage{color}
showspaces=false,            % показывать или нет пробелы специальными отступами
showstringspaces=false,      % показывать или нет пробелы в строках
showtabs=false,             % показывать или нет табуляцию в строках
frame=single,              % рисовать рамку вокруг кода
tabsize=2,                 % размер табуляции по умолчанию равен 2 пробелам
captionpos=t,              % позиция заголовка вверху [t] или внизу [b] 
breaklines=true,           % автоматически переносить строки (да\нет)
breakatwhitespace=false, % переносить строки только если есть пробел
escapeinside={\%*}{*)}   % если нужно добавить комментарии в коде
}

    \chapter{Обзор на CUDA}
    \section{Что такое CUDA?}
	
    CUDA (изначально аббр. от англ. Compute Unified Device Architecture) — программно-аппаратная архитектура параллельных вычислений, которая позволяет существенно увеличить вычислительную производительность благодаря использованию графических процессоров фирмы Nvidia.
    \begin{figure}[h!]
	\centering
	\includegraphics[scale=0.6]{b.png}
	\caption{Nvidia CUDA}
	\label{}
    \end{figure}
    \section{Формула расчета для индекса резьбы}
    Сетка может содержать несколько блоков, блоков блоков, может быть одномерным, двумерным или трехмерным. Блок содержит несколько потоков, которые также могут быть одномерными, двумерными или трехмерными.
    Каждый поток в CUDA имеет уникальный идентификатор идентификатора id-ThreadIDX, который варьируется с разницей в разделении сетки и блока, что дает формулу расчета идентификатора индекса резьбы в сетке и блокировать различный режим разделения.
    
    Сетка разделена на 1 размерность, а блок разделяется на 1 размерность.(\ref{index_rez})

    \begin{equation}
       	threadId = blockIdx.x *blockDim.x + threadIdx.x;  		
        \label{index_rez}
    \end{equation}

    

 \end{document}